\documentclass{report}


\input{preamble}
\input{macros}
\input{letterfonts}

\title{\Huge{Probabilidad} \\ Tarea 2}
\author{\huge{Omar Pacheco - Braulio Sanchez}}
\date{}

\begin{document}

\maketitle

\qs{}{
  La siguiente tabla de contingencia muestra el resultado (incompleto) de entrevistas
  concluidas durante una encuesta para estudiar las opiniones que se tienen al
  respecto del aborto legal en cierta ciudad.
  \begin{center}
    \begin{tabular}{|c|c|c|c|c|}
      \hline
      \textbf{Área de la ciudad} & \textbf{A favor (F)} & \textbf{En contra (Q)} & \textbf{Indecisos (R)} & \textbf{Total} \\ 
      \hline
      \textbf{A} & 100 & 20 & - & 125 \\
      \hline
      \textbf{B} & 115 & 5 & - & 125 \\
      \hline
      \textbf{D} & 50 & - & 15 & - \\
      \hline
      \textbf{E} & 35 & 50 & - & - \\
      \hline
      \textbf{Total} & - & 135 & 65 & - \\
      \hline
    \end{tabular}
  \end{center}
  Realice lo que se le pide proponiendo una notación adecuada para sus
  respuestas.

  \begin{enumerate}
    \item[a)] Si se elige al azar una de las personas encuestadas, cuál
      es la probabilidad de que:
      \begin{enumerate}
        \item[i. ] este en contra del aborto legalizado dado que vive en
          la zona A o E?
        \item[ii. ] El entrevistado esté indeciso dado que no vive en la
          zona D?
      \end{enumerate}
    \item[b)] Determine las siguientes probabilidades:
      \begin{enumerate}
        \item[i. ] $P(F|B)$
        \item[ii. ] $P(D|Q)$
        \item[iii. ] $P(Q| A \cup B)$
      \end{enumerate}
  \end{enumerate}
}

\qs{}{
  Supóngase que dividimos a las personas en tres clases económicas: pobre, media y
  rica. Mediante una encuesta, se encuentra que el 2\% de los pobres, el 43\% de la
  clase media, y el 70\% de los ricos tienen casa propia. Se sabe además que los pobres
  constituyen el 30\% de la población, la clase media el 68\%, los ricos el 2\%. AI
  seleccionar al azar una persona, se encuentra que tiene casa. ¿Cuál es la
  probabilidad de que esa persona sea de la clase pobre?, ¿Cuál es la probabilidad de
  que una persona seleccionada al azar en una población tenga casa propia?
}

\qs{}{
  Los miembros de una firma de consultoría rentan automóviles en tres agencias: 60%
  de la agencia 1, 30\% de la agencia 2 y 10\% de la agencia 3. Si 9\% de los vehículos de
  la agencia 1 necesitan afinación, 20\% de las unidades de la agencia 2 necesitan
  también una afinación y 6\% de los autos de la agencia 3 necesitan asimismo
  afinación, ¿cuál es la probabilidad de que un automóvil rentado a la firma necesite
  afinación?
}

\qs{}{
  En una fábrica, una línea de producción termina dos tipos de piezas ensambladas
  por dos autómatas. El primer autómata ensambla tres partes de la producción, y en
  65\% de los casos del acabado es de primera calidad. El segundo autómata ensambla
  dos partes de la producción, y en 85\% de los casos es de primera calidad.
  \begin{enumerate}
    \item[a) ] Si en esta línea se elige una pieza, cuál es la probabilidad de que sea
      de primera calidad?
    \item[b) ] Si una pieza es de primera calidad, ¿qué es más probable: que sea del primer
  automata o del segundo?
  \end{enumerate}
}

\newpage
\qs{}{
  Una máquina consiste en 4 componentes conectados en paralelo, de modo que la
  máquina falla sólo si los cuatro componentes fallan. Supongamos que los
  components son independientes entre sí. Si los componentes tienen probabilidades 
  0.1; 0.2; 0.3 y 0.4 de fallar cuando la máquina es encendida, ¿cuál es la probabilidad
  de que funcione al encenderla?
}

\qs{}{
  Se lanza un dado dos veces. Sea $A = \mathrm{"el\ máximo\ de\ las\ caras\ es\ 2"}$,
  $B = \mathrm{"el\ mímino\ de\ las\ caras\ es\ 2"}$. ¿Son independientes? Argumente su respuesta.
}

\qs{}{
  Si los tres primeros lanzamientos de una moneda han resultado cara, la probabilidad
  de que obtengamos cara en el cuarto lanzamiento es:
  \begin{enumerate}
    \item[A. ] $\dfrac{1}{16}$
    \item[B. ] $\dfrac{1}{4}$
    \item[C. ] $\dfrac{1}{2}$
    \item[D. ] Otro valor entre cero y uno
  \end{enumerate}
}

\qs{}{
  Una compañía maderera acaba de recibir un lote de 10,000 tablas de 2x4. Suponga
  que 20\% de estas tablas (2000) en realidad están demasiado tiernas o verdes para
  ser utilizadas en construcción de primera calidad. Se eligen dos tablas al azar, una
  después de la otra. Sea $A = \{ \mathrm{la\ primera\ tabla\ esta\ verde} \}$ y
  $B = \{ \mathrm{la\ segunda\ tabla\ está\ verde} \}$. Calcule 
  $P(A),\ P(B)\ \mathrm{y}\ P(A \cap B)$ (un diagrama de árbol podría ayudar).
  ¿Son A y B independientes?
}

\qs{}{
  Demuestre que para los eventos independientes $A$ y $B$ se tiene que
  \[P(A \cup B) = 1 - P(A^c)P(B^c)\]
  
  Generalización (inducción): Si los eventos $A_1, A_2, \dots, A_n$ son
  mutuamente indepndientes, entonces
  \[P(A_1 \cup A_2 \cup \dots \cup A_n) = 1 - P(A_1^c)P(A_2^c) \dots P(A_n^c).\]
}

\qs{}{
  Si $A$ y $B$ son eventos independientes y la probabilidad de que ambos ocurran es $0.16$,
  mientras que la probabilidad de que ninguno ocurra es $0.36$, calcule $P(A)$ y $P(B)$.
}

\pf{Demostración del inciso 1.1}{
  Es fácil ver que la pregunta 1 es verdad.
}

\end{document}
