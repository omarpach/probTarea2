\documentclass{report}

\input{preamble}
\input{macros}
\input{letterfonts}

\title{\Huge{Probabilidad} \\ Tarea 2}
\author{\huge{Omar Pacheco - Braulio Sanchez}}
\date{}

\begin{document}

\maketitle

\qs{}{
  La siguiente tabla de contingencia muestra el resultado (incompleto) de entrevistas
  concluidas durante una encuesta para estudiar las opiniones que se tienen al
  respecto del aborto legal en cierta ciudad.
  \begin{center}
    \begin{tabular}{|c|c|c|c|c|}
      \hline
      \textbf{Área de la ciudad} & \textbf{A favor (F)} & \textbf{En contra (Q)} & \textbf{Indecisos (R)} & \textbf{Total} \\ 
      \hline
      \textbf{A} & 100 & 20 & - & 125 \\
      \hline
      \textbf{B} & 115 & 5 & - & 125 \\
      \hline
      \textbf{D} & 50 & - & 15 & - \\
      \hline
      \textbf{E} & 35 & 50 & - & - \\
      \hline
      \textbf{Total} & - & 135 & 65 & - \\
      \hline
    \end{tabular}
  \end{center}
  Realice lo que se le pide proponiendo una notación adecuada para sus
  respuestas.

  \begin{enumerate}
    \item[a)] Si se elige al azar una de las personas encuestadas, cuál
      es la probabilidad de que:
      \begin{enumerate}
        \item[i. ] este en contra del aborto legalizado dado que vive en
          la zona A o E?
        \item[ii. ] El entrevistado esté indeciso dado que no vive en la
          zona D?
      \end{enumerate}
    \item[b)] Determine las siguientes probabilidades:
      \begin{enumerate}
        \item[i. ] $P(F|B)$
        \item[ii. ] $P(D|Q)$
        \item[iii. ] $P(Q| A \cup B)$
      \end{enumerate}
  \end{enumerate}
}


\pf{Demostración del inciso 1.1}{
  Es fácil ver que la pregunta 1 es verdad.
}

\end{document}
